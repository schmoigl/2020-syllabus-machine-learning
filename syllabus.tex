\documentclass[11pt, a4paper]{article}
%\usepackage{geometry}
\usepackage[inner=1.5cm,outer=1.5cm,top=2.5cm,bottom=2.5cm]{geometry}
\pagestyle{empty}
\usepackage{graphicx}
\usepackage{fancyhdr, lastpage, bbding, pmboxdraw}
\usepackage[usenames,dvipsnames]{color}
\definecolor{darkblue}{rgb}{0,0,.6}
\definecolor{darkred}{rgb}{.7,0,0}
\definecolor{darkgreen}{rgb}{0,.6,0}
\definecolor{red}{rgb}{.98,0,0}
\usepackage[colorlinks,pagebackref,pdfusetitle,urlcolor=darkblue,citecolor=darkblue,linkcolor=darkred,bookmarksnumbered,plainpages=false]{hyperref}
\renewcommand{\thefootnote}{\fnsymbol{footnote}}

\pagestyle{fancyplain}
\fancyhf{}
\lhead{ \fancyplain{}{Data Science and Machine Learning} }
%\chead{ \fancyplain{}{} }
\rhead{ \fancyplain{}{\today} }
%\rfoot{\fancyplain{}{page \thepage\ of \pageref{LastPage}}}
\fancyfoot[RO, LE] {page \thepage\ of \pageref{LastPage} }
\thispagestyle{plain}

%%%%%%%%%%%% LISTING %%%
\usepackage{listings}
\usepackage{caption}
\DeclareCaptionFont{white}{\color{white}}
\DeclareCaptionFormat{listing}{\colorbox{gray}{\parbox{\textwidth}{#1#2#3}}}
\captionsetup[lstlisting]{format=listing,labelfont=white,textfont=white}
\usepackage{verbatim} % used to display code
\usepackage{fancyvrb}
\usepackage{acronym}
\usepackage{amsthm}
\VerbatimFootnotes % Required, otherwise verbatim does not work in footnotes!



\definecolor{OliveGreen}{cmyk}{0.64,0,0.95,0.40}
\definecolor{CadetBlue}{cmyk}{0.62,0.57,0.23,0}
\definecolor{lightlightgray}{gray}{0.93}



\lstset{
%language=bash,                          % Code langugage
basicstyle=\ttfamily,                   % Code font, Examples: \footnotesize, \ttfamily
keywordstyle=\color{OliveGreen},        % Keywords font ('*' = uppercase)
commentstyle=\color{gray},              % Comments font
numbers=left,                           % Line nums position
numberstyle=\tiny,                      % Line-numbers fonts
stepnumber=1,                           % Step between two line-numbers
numbersep=5pt,                          % How far are line-numbers from code
backgroundcolor=\color{lightlightgray}, % Choose background color
frame=none,                             % A frame around the code
tabsize=2,                              % Default tab size
captionpos=t,                           % Caption-position = bottom
breaklines=true,                        % Automatic line breaking?
breakatwhitespace=false,                % Automatic breaks only at whitespace?
showspaces=false,                       % Dont make spaces visible
showtabs=false,                         % Dont make tabls visible
columns=flexible,                       % Column format
morekeywords={__global__, __device__},  % CUDA specific keywords
}

%%%%%%%%%%%%%%%%%%%%%%%%%%%%%%%%%%%%
\begin{document}
\begin{center}
{\Large \textsc{Data Science and Machine Learning}}
\end{center}
\begin{center}
Fall 2020
\end{center}
%\date{September 26, 2014}

\begin{center}
\rule{6in}{0.4pt}
\begin{minipage}[t]{.75\textwidth}
\begin{tabular}{llcccll}
\textbf{Instructors:} & Thomas Mitterling, Max Thomasberger, Lukas Schmoigl \\
\textbf{Email:} &  \href{mailto:lukas.schmoigl@wifo.ac.at}{lukas.schmoigl@wifo.ac.at} \\
\textbf{Mode:} & 3 hours lecture + 2 hours research seminar \\
\textbf{Time:} & tbd \\
\textbf{Place:} & tbd \\
\end{tabular}
\end{minipage}
\rule{6in}{0.4pt}
\end{center}
\vspace{.5cm}
\setlength{\unitlength}{1in}
\renewcommand{\arraystretch}{2}

\noindent\textbf{Course Pages:} \begin{itemize}
\item \url{https://learn.wu.ac.at/}
\end{itemize}

\vskip.15in
\noindent\textbf{Office Hours:} After class, or by appointment, or post your questions in the forum provided for this purpose on LEARN.

\vskip.15in
\noindent\textbf{Main References:} %\footnotemark
This is a  restricted list of various interesting and useful books that will be touched during the course. You need to consult them occasionally.
\begin{itemize}
\item Christopher M. Bishop, {\textit{Pattern Recognition and Machine Learning}}, Springer, 2006.
\item Richard O. Duda, Peter E. Hart, and David G. Stork, {\textit{Pattern Classification}}, Wiley, 2nd ed., 2000.
\item Peter Flach, {\textit{Machine Learning: The Art and Science of Algorithms that Make Sense of Data}}, Cambridge University Press, 2012.

\end{itemize} 

% \footnotetext{.}

\vskip.15in
\noindent\textbf{Objectives:}  This course is designed for graduate students of economics and offers an introduction into modern techniques in data science. 

\vskip.15in
\noindent\textbf{Prerequisites:}
Basic understanding of probability, statistics, linear algebra and calculus is assumed. Programming skills are required (e.g. R, Python,...)


\vspace*{.15in}

\noindent \textbf{Tentative Course Outline:}
\begin{center} 
\begin{minipage}{5in}
\begin{flushleft}
%Chapter 1 \dotfill ~$\approx$ 3 days \\
{\color{darkgreen}{\Rectangle}} ~Machine Leraning 1: Regression and classification trees \\
{\color{darkgreen}{\Rectangle}} ~Machine Learning 2: Cross-Validation and model validation techniques \\
{\color{darkgreen}{\Rectangle}} ~Machine Learning 3: Document classification \\
{\color{blue}{\Rectangle}} ~Data Science 1: Data wrangling \\
{\color{blue}{\Rectangle}} ~Data Science 2: Data science toolkit (e.g. APIs, geocoding, fuzzy matching,...) \\
{\color{blue}{\Rectangle}} ~Data Science 3: Data visualization \\
\end{flushleft}
\end{minipage}
\end{center}

\vspace*{.15in}
\noindent\textbf{Grading Policy:} lecture: Midterm (40\%), Final Exam (40\%), class participation (20\%), research seminar: Homework (50\%), Project (50\%).

\vskip.15in
\noindent\textbf{Important Dates:}
\begin{center} \begin{minipage}{3.8in}
\begin{flushleft}
Midterm      \dotfill ~tbd  \\
Final Exam       \dotfill ~tbd  \\
Project Deadline \dotfill ~tbd \\

\end{flushleft}
\end{minipage}
\end{center}

\vskip.15in
\noindent\textbf{Class Policy:}  
\begin{itemize}
\item Regular attendance is essential and expected.
\end{itemize}

%%%%%% THE END 
\end{document} 